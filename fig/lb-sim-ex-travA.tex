\begin{figure}[t]
$\hfill\inarr{\begin{tikzpicture}[xscale=1,yscale=1.5]
  \node (init) at (2,  1)   {$\Init$};
  \node (i11)  at (0,  0)   {$\mese{1}{1}{} \rlab{}{x}{1}$};
  \node (i12)  at (0, -1)   {$\mese{1}{2}{} \wlab{}{y}{1}$};
  \node (i13)  at (0, -2)   {$\mese{1}{3}{} \wlab{}{z}{1}$};
  \node (i21)  at (4,  0)   {$\mese{2}{1}{} \rlab{}{y}{1}$};
  \node (i22)  at (4, -1)   {$\mese{2}{2}{} \rlab{}{z}{1}$};
  \node (i23)  at (4, -2)   {$\mese{2}{3}{} \wlab{}{x}{1}$};
  %% \node (hh)   at (2, -3.5) {$\inarrC{\text{The execution graph } G \text{ and} \\\text{its traversal configuration } \TCa}$};
  \begin{scope}[on background layer]
     \issuedCoveredBox{init};
     \issuedBox{i13};
%     \issuedBox{i23};
  \end{scope}
  \draw[rf] (i13) edge node[above] {} (i22);
  \draw[rf] (i23) edge node[above] {} (i11);
  \draw[rf] (i12) edge node[above] {} (i21);
% \draw[vf] (init) edge[bend right=20]  node[above left, pos=0.9] {$\lVF$} (i11);
% \draw[vf] (init) edge[bend left=20]  node[above right, pos=0.9] {$\lVF$} (i21);
% \draw[vf] (i13)  edge[bend right=20] node[above] {$\lVF$} (i22);
  \draw[ppo,out=230,in=130] (i11) edge node[left ,pos=0.8] {\small$\lPPO$} (i12);
  \draw[ppo,out=310,in=50 ] (i22) edge node[right,pos=0.3] {\small$\lPPO$} (i23);
  \draw[po] (init) edge (i11);
  \draw[po] (init) edge (i21);
  \draw[po] (i11)  edge (i12);
  \draw[po] (i12)  edge (i13);
  \draw[po] (i21)  edge (i22);
  \draw[po] (i22)  edge (i23);
\end{tikzpicture}}
\hfill\vrule\hfill
\inarr{\begin{tikzpicture}[scale=0.8, every node/.style={transform shape}]

  \node (init) at (0, 1)   {$\Init$};

  \node (i111) at (-1.5,  0)   {$\mese{1}{1}{1} \rlab{}{x}{0}$};
  \node (i121) at (-1.5, -1)   {$\mese{1}{2}{1} \wlab{}{y}{0}$};
  \node (i131) at (-1.5, -2)   {$\mese{1}{3}{1} \wlab{}{z}{1}$};

  \node (i211) at (0.5,  0)   {\phantom{$\mese{2}{1}{1} \rlab{}{y}{0}$}};
  \node (i221) at (0.5, -1)   {\phantom{$\mese{2}{2}{1} \rlab{}{z}{1}$}};
  \node (i231) at (0.5, -2)   {\phantom{$\mese{2}{3}{1} \wlab{}{x}{1}$}};

  \draw[jf] (init) edge[bend right] node[above]        {\small{$\lJF$}} (i111);

  \draw[po] (init)  edge (i111);
  \draw[po] (i111)  edge (i121);
  \draw[po] (i121)  edge (i131);

  \begin{scope}[on background layer]
    \draw[extractStyle] (-3, 1.5) rectangle (1,-2.5);
  \end{scope}

  %% \node (hh) at (0, -3.5) {$\inarrC{\text{The event structure } \ESa \text{ and} \\\text{the selected execution } \SXa}$};
\end{tikzpicture}}\hfill$
\caption{%
Граф сценария исполнения $G$, 
конфигурация его обхода $\TC_a$
и соответствующая этой конфигурации
структура событий $S_a$ вместе с конфигурацией $X_a$.
Покрытые события выделены как 
{\protect\tikz \protect\draw[coveredStyle] (0,0) rectangle ++(0.35,0.35);}
, а выпущенные как
{\protect\tikz \protect\draw[issuedStyle] (0,0) rectangle ++(0.35,0.35);}.
События, принадлежащие конфигурации $X_a$, выдены как 
{\protect\tikz \protect\draw[extractStyle] (0,0) rectangle ++(0.35,0.35);}.
}
\label{fig:lb-sim-ex-travA}
\end{figure}
