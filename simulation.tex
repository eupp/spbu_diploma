\pagebreak

\section{Симуляция обхода графа \IMM}
\label{sec:simulation}

В данном разделе приводится более подробный обзор
процесса симуляции построения структуры событий по обходу \IMM графа.
В разделе \ref{sec:imm-trav} детально рассматривается
операционная семантика обхода графа \IMM.
В разделе \ref{sec:simrel} описывается
отноешние симуляции $\simrel$.
Наконец, в разделе \ref{sec:simstep} на примерах разбирается
процесс симуляции шага обхода графа
семантикой построения структуры событий. 

\subsection{Обход графа \IMM}
\label{sec:imm-trav}

Определение обхода графа \IMM. Основные инварианты обхода.

\subsection{Отношение симуляции}
\label{sec:simrel}

Далее опишем отношение симуляции $\simrel$.
В целях ясности и простоты изложения, 
в данном разделе будет приведена упрощенная версия формального
определения этого отношения, которая опускает некоторые
технические детали. Полная версия отношения симуляции
может быть найдена в \coq репозитории. 

Отношение симуляции $\simrel(P, T, G, TC, S, X)$
устанавливает взаимосвязь между структурой событий $S$
и графом сценария исполнения $G$ с помощью
функции $\ea : S.\lE \fun G.\lE$, которая отображает
события структуры $S$ в события графа $G$.
Эта функция может быть натуральным образом
расширена на множества событий следующим образом%
\footnote{аналогично образом функций $\ea$ может быть расширена
на бинарные отношения на событиях.}:

\begin{align*}
\text{for } A_S \subseteq S.\lE        & :
  \fmap{A_S} \defeq \set{\ea(e) \in G.\lE \mid e \in A_S} \\
\text{for } A_G \subseteq G.\lE        & :
  \fcomap{A_G} \defeq \set{e \in S.\lE \mid \ea(e) \in A_G}.
\end{align*}

Отношение симуляции $\simrel(P, T, G, \TC, S, X)$ состоит из следующих свойств.

\begin{enumerate}

  \item \label{simrel:events}
    События $S$, принадлежащие потокам из $T$, а также события,
    принадлежащие конфигурации $X$, соответствуют покрытым событиям,
    а также выпущенным событиям и их $\lPO$-предшественникам: 
    \begin{itemize}
      \item $\fmap{S.\lE\rst{T}} = \fmap{X} = C \cup \dom{G.\lPO^? \seq [I]}$
    \end{itemize}

  \item \label{simrel:lab}
    Метки событий из $S$ совпадают с метками событий из $G$
    по модулю прочитанных или записанных значений. 
    \begin{enumerate}
      \setcounter{enumii}{0}
      \item \label{simrel:lab-eqmval}
        $\forall e \in S.\lE \ldotp\;
          S.\set{\lTID, \lTYP, \lLOC, \lMOD}(e) =
          G.\set{\lTID, \lTYP, \lLOC, \lMOD}(\fmap{e}) $
    \end{enumerate}
    Метки покрытых и выпущенных событий, принадлежащих конфигурации $X$,
    сопадают полностью.
    \begin{enumerate}
      \setcounter{enumii}{1}
      \item \label{simrel:lab-det}
        $\forall e \in X \cap \fcomap{C \cup I} \ldotp~
          S.\lVAL(e) = G.\lVAL(\ea(e))$
    \end{enumerate}

  \item \label{simrel:po}
    Программный порядок в структуре событий $S$
    совпадает с программным порядком в графе $G$:
    \begin{itemize}
      \item $\fmap{S.\lPO} \suq G.\lPO$
    \end{itemize}

  \item \label{simrel:cf}
    Если два события имеют одинаковый образ под действием функции $\ea$,
    то эти события равны или находятся в конфликте.
    \begin{itemize}
      \item $\fcomap{\mathtt{id}} \suq S.\lCF^?$
    \end{itemize}

  \item \label{simrel:jf}
    События чтения в $S$ должны быть обоснованы событиями записи,
    которые наблюдаются соответствующим событием чтением в $G$.
    \begin{enumerate}
      \item \label{simrel:jf-obs}
      \setcounter{enumii}{0}
        $\fmap{S.\lJF} \suq G.\lRF^?\seq G.\lHB^?$
    \end{enumerate}
    Более того, отношение $\lJF$, ограниченное на события чтения,
    принадлежащие конфигурации $X$, соответствуют
    отношению \emph{стабильной обоснованности} (\emph{stable justification})
    (смотри \cref{def:sjf}) в графе $G$.
    \begin{enumerate}
      \setcounter{enumii}{1}
      \item \label{simrel:jf-sjf}
        $\fmap{S.\lJF \seq [X]} \suq G.\lSRF_{TC}$
    \end{enumerate}
    %% As a consequence it is possible to derive that
    %% justification for covered events in $X$
    %% corresponds to their justification in $G$:
    %% $\fmap{S.\lJF \seq [X \cap \fcomap{C}]} \subseteq G.\lRF$. \\
    Только выпущенные события могут быть использованы для
    внешнего обоснования событий чтения. 
    \begin{enumerate}
      \setcounter{enumii}{2}
      \item \label{simrel:jfe-iss}
         $\dom{S.\lJFE} \suq \dom{S.\lEW \seq [X \cap \fcomap{I}]}$
    \end{enumerate}

  \item \label{simrel:ew}
    Все эквивалентные события записи в $S$ отображаются
    в одну и то же событие записи $G$.
    \begin{enumerate}
      \setcounter{enumii}{0}
      \item \label{simrel:ew-id}
        $\fmap{S.\lEW} \suq \mathtt{id}$
    \end{enumerate}
    Также. каждый класс эквивалентности по отношению $S.\lEW^*$
    должен иметь представителя среди выпущенных событий,
    принадлежащих конфигурации $X$.
    \begin{enumerate}
      \setcounter{enumii}{1}
      \item \label{simrel:ew-iss}
        $S.\lEW \suq (S.\lEW \seq [X \cap \fcomap{I}] \seq S.\lEW)^?$
    \end{enumerate}

  \item \label{simrel:co}
    Если два события структуры $S$ находящихся в отношении когерентности,
    то их образы под действием функции либо также находятся
    в отношении когерентности, либо равны. 
    \begin{enumerate}
      \setcounter{enumii}{0}
      \item \label{simrel:co-co}
         $\fmap{S.\lCO} \suq G.\lCO^?$
    \end{enumerate}
    Если же ребро отношения когерентности оканчивается
    в событие, принадлежащем конфигурации $X$ и одному из потоков из $T$,
    тогда образ этого ребра принадлежит отношению когерентности в графе $G$.
    \begin{enumerate}
      \setcounter{enumii}{1}
      \item \label{simrel:co-cfg}
         $\fmap{S.\lCO \seq [X\rst{T}]} \suq G.\lCO$
    \end{enumerate}

  \item \label{simrel:sw-hb}
    Отношения ``синхронизируется-с'' и ``происходит-до''
    в структуре событий $S$ согласованы с соответствующими
    отношениями в графе $G$.
    \begin{enumerate}
      \item \label{simrel:sw}
        $\fmap{S.\lSW} \suq G.\lSW$
      \item \label{simrel:hb}
        $\fmap{S.\lHB} \suq G.\lHB$
    \end{enumerate}

\end{enumerate}


\subsection{Шаг симуляции}
\label{sec:simstep}

Описание шага симуляции. Процесс добавление новых событий чтения и записи.
Обоснование сохранения отношения симуляции при добавлении
нового события в структуру событий.

\begin{definition}
\label{def:sjf}
Отношение \emph{стабильной обоснованности}
определяется следующим образом.
\begin{equation*}
  G.\lSRF_{TC} \defeq
    ([G.\lW] \seq (G.\lVF_{TC} \cap {=}_{G.\lLOC}) \seq [G.\lR])
    \setminus (G.\lCO \seq G.\lVF_{TC})
\end{equation*}
\end{definition}

%% \subsection{Отношение симуляции для сертификационной ветки}

%% Определение отношения симуляции для сертификации. Основные инварианты этого отношения.
%% \todo{Этот раздел опционален.}
